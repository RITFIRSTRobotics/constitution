\documentclass[english,11pt]{article}

%% packages
\usepackage{fancyhdr} % header and footer commands
\usepackage[margin=.6in]{geometry} % sets the margins
\usepackage[useregional]{datetime2} % \today
\usepackage{enumitem} % used for ordered lists
\usepackage{lmodern} % use high-quality fonts

%% setup the footers
\pagestyle{fancy}
\fancyhf{}
\lfoot{RIT FIRST Alumni Association}
\rfoot{Page \thepage}
\renewcommand{\headrulewidth}{0pt} % disable header line
\renewcommand{\footrulewidth}{1pt} % set thin footer line

%% Redefine sections to be "Articles"
% see https://tex.stackexchange.com/questions/136527/section-numbering-without-numbers
\makeatletter
% we use \prefix@<level> only if it is defined
\renewcommand{\@seccntformat}[1]{%
  \ifcsname prefix@#1\endcsname
    \csname prefix@#1\endcsname
  \else
    \csname the#1\endcsname\quad
  \fi}
% define \prefix@section
\newcommand\prefix@section{Article \thesection: }
\newcommand\prefix@subsection{Section \thesubsection: }
\makeatother

%% setup the draft watermark, if applicable
\input{watermark.tex} % autogenerated file that will contain watermark info if it is needed

%% start the document
\begin{document}

%% build the title page
\title{Constitution for the FIRST Robotics Club of the Rochester Institute of Technology}
\author{
    Brett Johnson, Lorenzo Romero, Dustyn Crowley, Connor Henley, et al.  % your name here (before et al.) <<<<<
    \\
    \texttt{[btj3571, llr4395, dtc3536, cxh1451, first]@rit.edu}
}
\date{} % don't put the date here

% setup the footer for the title page
\fancypagestyle{plain}{
    \fancyhf{}
    \input{buildinfo.tex} % take the lfoot from the autogenerated build info file
    \rfoot{\today}
    \renewcommand{\footrulewidth}{0pt} % disable footer line
}

\maketitle
\newpage % finish title page

% notes:
%   * it is best practice to have a sentence per line, as git will work better
%   * try to use the existing packages/styling

%% start constitution
\section*{Preface}
This document is the official constitution of the FIRST Robotics club, herein referred to as the Club, at the Rochester Institute of Technology, herein referred to as the University.
The guidelines and policies laid out in the following text shall be adhered to, to the best of their ability, by all members of Club Administration, and members of the club at large.

% Article: name and formal stuff
\section{Formal Name} \label{art:formal-name}
The name of this organization shall be the FIRST Robotics Club for all business within the University.
For business outside of the university, the organization shall do business as the RIT FIRST Alumni Association (formally) or RIT FIRST (informally).

% Article: mission
\section{Organization Mission} \label{art:mission}
The mission of the Club is to promote FIRST (For Inspiration and Recognition of Science and Technology) robotics, support and foster interest in robotics programs in both the local community and in general, and provide support for local teams, which includes hosting events, running programs to help teams grow, and maintaining their interest in science and technology.

% Article: membership
\section{Membership} \label{art:membership}

\subsection{Requirements for Membership} \label{sect:membership:requirements}
Membership is open to all regularly enrolled undergraduate and graduate students, alumni, faculty, staff, and part time students in good standing with the University regardless of their previous experience in FIRST or a lack thereof.

\subsection{Definition of Active Membership} \label{sect:membership:active}
Active members shall be those who meet the following criteria:
\begin{itemize}
    \item regularly enrolled undergraduate or graduate student, including those on local co-ops if they so choose,
    \item meets the following attendance requirements:
    \begin{itemize}
        \item maintains a 70\% attendance rate, where the attendance rate is defined as the number of weeks that the member attended one or more club sanctioned meeting or activity, divided by the number of weeks completed in the semester since the first meeting the member attended,
        \item was either an Active Member in the previous term or has attended at least four meetings this term,
    \end{itemize}
    \item paid club dues to date (if required),
    \item met the qualifications set forth in Section \ref{sect:membership:requirements} along with those of the Center for Campus Life.
\end{itemize}

\subsection{Definition of Associate Membership} \label{sect:membership:associate}
Associate members shall be those who have interest in the club but no interest in or availability to participate.
Associate members may attend meetings and discussions, but shall have no power to vote in the organization and cannot hold office.

\subsection{Dues} \label{sect:membership:dues}
Dues may be set by the Club Administration at any point to provide additional funding for club activities.
If dues are set, all active members must be notified three regular meetings in advance.

\subsection{Definition of Inactive Membership} \label{sect:membership:inactive}
An Active Member shall be declared Inactive if they have missed the required number of meetings or more, as defined in Section \ref{sect:membership:active}.
Inactive members share the same privileges and restrictions as Associate Members.

\subsection{Temporary Inactive Status} \label{sect:membership:tmp-inactive}
Members can request temporary inactive status for the duration of the semester if they will be unable to attend regular club meetings (e.g. member will be on co-op or cannot make meetings due to a class, etc.).
Requests must be submitted to the President or Vice President and will be approved at the following Club Administration regular meeting.
Members with temporary inactive status have the same rights as Inactive Members.

\subsection{Member Resignation} \label{sect:membership:resignation}
A member may resign by submitting the intention, in writing, to the President, which becomes effective as soon as it is submitted.

\subsection{Dismissal from Club Roster} \label{sect:membership:dismissal}
A member who has missed more than eight meetings within a semester without notification shall be dropped from the club roster and removed from all club resources (dismissed).

% Article: meetings
\section{Meetings} \label{art:meetings}

\subsection{Decision of Meeting Time} \label{sect:meetings:decision}
Meeting times shall be decided by members of the Club Administration and must be done three weeks before the beginning of the next semester.
The new time shall be decided with consideration of the availability of all members who respond to a request for available times.

\subsection{Announcement of Meeting Time} \label{sect:meetings:announcement}
Meeting times shall be posted to all Club communication channels and social media platforms before the start of the next academic term.

\subsection{Attendance of Meetings} \label{sect:meetings:attendance}
Attendance at any regular meeting is open to persons who meet the criteria defined in Section \ref{sect:membership:requirements}.

% Article: definition of club administration and how it functions
\section{Club Administration Roles and Selection} \label{act:cadmin}
\subsection{President} \label{sect:cadmin:president}
The President is the ultimate presiding figure for the Club.
The responsibilities and duties of the President are as follows:

\begin{enumerate}[label=\Alph*.]
    \item Preside over all club meetings and Club Administration meetings.
    \item Exercise general supervision over all club activities.
    \item Give final approval on any club expense.
    \item Support club administrators and club members in fulfilling the Club Mission Statement
    \item Ensure that appropriate members of Club Administration are financially certified with the club office.
\end{enumerate}

\subsection{Vice President} \label{sect:cadmin:vicepresident}
The responsibilities and duties of the Vice President are as follows:

\begin{enumerate}[label=\Alph*.]	
    \item Assume all responsibilities and duties of the President during the temporary absence of the President.
    \item Shall be responsible to maintain open channels of communication between the Club and mentored teams/organizations.
    \item Oversee all ad hoc administrative roles.
\end{enumerate}

\subsection{Treasurer} \label{sect:cadmin:treasurer}
The responsibilities and duties of the Treasurer are as follows:

\begin{enumerate}[label=\Alph*.]
    \item Submit a monthly report of club finances to the President of Vice President at Club Administration meetings.
    \item Fill out, at the request of the President or Vice President, Expense Approval Forms (EAFs) and any other required financial documents for club events and projects.
    \item Accept all requests for purchases or reimbursements from members of the Club, and get approval from the President or Vice President.
    \item Assume the duties and responsibilities of the Vice President in their absence.
    \item Supervise all financial transactions of the organization and maintain all financial records.
    \item Be responsible for communication with the Director of Finance of the Center for Campus Life.
\end{enumerate}

\subsection{Media and Communications Manager} \label{sect:cadmin:mediacomms}
The responsibilities and duties of the Media and Communications Manager are as follows:

\begin{enumerate}[label=\Alph*.]
    \item Update, or oversee updates to, news sections on the Club website.
    \item Coordinate, at the request of the President or Vice President, distribution of flyers and other club promotional materials.
    \item Act as a liaison between the Club and FIRST community regarding club events, club presence at local competitions, and mentoring.
    \item Actively seek out, at the request of the President or Vice President, new opportunities for mentor-ship and club involvement in the local FIRST community.
    \item Maintain an active social media presence via the club social media accounts, such as periodic posts about recent meetings and project milestones, and reminders about upcoming meetings and events.
\end{enumerate}

\subsection{Ad Hoc Administrative Positions} \label{sect:cadmin:adhoc}
The following positions exist as needed, and may be filled at the will of the President by a club member, including one already in a Club Administration position, should they accept the appointment:

\begin{enumerate}[label=\Alph*.]
    \item {\large Data Steward} \\
            The Data Steward shall be a member of the club who is well versed in the operations of the Club's document and resource storage system and be able to effectively administer said system.
            The Data Steward shall also maintain the Club's project management system, and collaboration sites.
    \item {\large Webmaster} \\
            The Webmaster shall maintain the web presence of the Club, primarily through the official club website (ritfirst.org).
            They shall fulfill requests of the President and Vice President regarding content posted on the official Club website.
            It will be the responsibility of the Webmaster to assist club members in getting relevant information available on the web, including project updates, meeting times, and other club resources.
            The Webmaster shall also maintain the Club's communication channels.
    \item {\large Imagine RIT Project Manager} \\
            The Imagine Project Manager shall assume ultimate responsibility for the completion of the Imagine project in a timely manner, planning and logistics, processing orders related to the project, and managing project resources.
    \item {\large FRC Robot Coordinator} \\
            The FRC Robot Coordinator shall oversee any projects regarding FRC robots obtained from local schools.
            They shall ensure that all components and parts obtained are inventoried, and project resources are managed properly.
    \item {\large Meeting Secretary} \\
            The Meeting Secretary shall compile meeting minutes and send them out the club at-large via email.
            In the event that the Meeting Secretary will be absent, they must notify the President and suggest another member to fulfil the role.
\end{enumerate}

\subsection{Selection of Club Administration Roles} \label{sect:cadmin:selection}

\subsubsection{Qualifications for Candidate} \label{subsect:cadmin:selection:qualifications}
Any candidate for a Club Administration role must be in good academic standing as defined by the University.
The candidate must have been an active member in good standing in the club during the previous academic term, not including any summer terms.

\subsubsection{Eligibility to Vote} \label{subsect:cadmin:selection:eligibility}
In order to vote, members must meet the Active Member qualifications set forth in Section \ref{sect:membership:active}.

\subsubsection{Process for Election} \label{subsect:cadmin:selection:election}
An election must take place at least 2 meetings prior to the last Club meeting for that academic year. The following guidelines must be followed:

\begin{enumerate}[label=\alph*.]
    \item Nominations shall be made for members from an open floor at the meeting prior to the election.
          A person may be nominated for more than one position. The nominee must meet the qualifications set forth for Active Members in Section \ref{sect:membership:active}.
    \item Members who will be on co-op during their term may not be nominated for the role of President, unless there are no eligible members that accept nominations, or for the role of Vice President, unless the Club Administration unanimously approves or there are no eligible members that accept nominations.
          However, both the President and Vice President cannot be on co-op at the same time.
          Members who are on co-op during nominations and will be eligible for a Club Administration position when they return, can be nominated and submit their speech to the President, who will distribute it (electronically or physically) to all eligible voting club members.
    \item The order of election shall be President, Vice President, Treasurer, and Media \& Communications
    \item Votes shall be cast by secret ballot and counted by the President.
    \item Elections shall require a majority vote of the present Active Members.
          If the majority of voting members abstain from the vote or hand in blank votes, then a session of nominations will start immediately, where any Active Member can be nominated for the contested role, each nominee will be eligible to give a speech if they have not already given one, and a vote consisting of all previous and new nominees will commence.
          This process will repeat until a member is elected for this position.
    \item Inductions shall not take place until after a meeting between the current Club Administration and incoming Club Administration.
    \item At the aforementioned meeting, the incoming Club Administration shall be presented with all information pertaining to the current state of the club.
          This meeting should also include all important persons that frequently deal with the club, such as the club advisor(s).
\end{enumerate}

\subsubsection{Term of Service} \label{subsect:cadmin:selection:terms}
Members of Club Administration shall serve from the last day of the previous academic year to the last day of the academic year for which they were elected.

\subsection{Resignation or Impeachment of a Club Administration Member} \label{sect:cadmin:removal}

\subsubsection{Impeachment of a Club Administration Member} \label{subsect:cadmin:removal:impeachment}

\begin{enumerate}[label=\alph*.]
    \item Any member of the Club Administration may be impeached.
    \item Ad hoc roles may be dismissed by order of the President.
    \item Impeachment may be initiated by petition, in writing, by 25\% or more of the Club's Active Members, when presented to the floor in a regular meeting.
    \item At the next meeting, the accuser (members signing the petition) and the accused (club administrator) shall present their case to the members.
        This meeting shall be presided over by the Club Advisor.
    \item Both cases will have the opportunity to present an introduction, an argument, and a conclusion (in this order).
        The impeachment trial will start with the accuser presenting their introduction, then the accused will present their introduction; this alternating process will continue for the arguments and conclusions.
        Each case is limited to a total of 20 minutes of presentation time, each group can use it as they see fit.
    \item After the trial is completed, a secret ballot vote shall be taken and counted by the Club Advisor.
    \item Conviction of impeachment shall require 2/3 vote by present Active Members.
        In the event of a tied vote, the Club Advisor shall break the tie.
    \item Conviction of impeachment shall cause immediate removal from office, loss of all privileges thereof, and suspension of all responsibilities and duties that role entails.
    \item The filling of a position vacated by impeachment shall be by a special election held in the manner of the annual elections.
        Nominations for this election shall be held at the meeting following impeachment.
        The special election shall be held at the meeting following nominations.
\end{enumerate}

\subsubsection{Resignation of a Club Administration Member} \label{subsect:cadmin:removal:resignation}

\begin{enumerate}[label=\alph*.]
    \item Any member of the Club Administration may resign at his/her discretion.
    \item The resignation letter must be presented to the club in the same manner as a member resignation as outlined in Section \ref{sect:membership:resignation}.
    \item The filling of a vacancy as a result of a resignation will be handled in the same manner as an impeachment.
\end{enumerate}

\subsection{Other Guidelines for Club Administration Members} \label{subsect:cadmin:other}

\begin{enumerate}[label=\Alph*.]
    \item The members of Club Administration shall be ranked in the following order: President, Vice President, Treasurer, Media \& Communications Manager.
    \item No member of Club Administration can hold more than one non-ad hoc position at any given time.
    \item All members of Club Administration, except for those who are only filling an ad hoc position, shall be financially certified through the Center for Campus Life.
    \item If a member of the Club Administration goes on co-op or otherwise becomes an Inactive Member during their term, the vacancy created will be filled in the same manner as an impeachment.
          This elected member's term will continue from election until the original member returns from co-op or the term expires.
          This elected member will effectively replace the original member in the Club Administration, except in Club Administration communications channels, where both members may remain.
    \item If the club is participating in College of Engineering Pizza Sales, then all Club Administration members must complete pizza sale training, as specified by the College.
\end{enumerate}

% Article: decision making
\section{Administration Decision Making Process} \label{art:decision-making}

\subsection{Process for a Voted Decision} \label{sect:decision-making:voted}
The following process shall be followed for a voted decision:

\begin{enumerate}[label=\alph*.]
    \item the President must announce the options which can be voted for
    \item ballots must be distributed to all eligible voting members present
    \item the President shall then collect the complete ballots from members
    \item the President shall then count the ballots and the option with a majority vote is considered the winning result of the election
\end{enumerate}

\subsection{Process for an Executive Order} \label{sect:decision-making:executive-order}
Executive Orders are decisions made exclusively by the President.
Executive Orders do not require a vote by the general club nor the Club Administration.

\subsection{Process for Overriding a Decision} \label{sect:decision-making:override}
\begin{enumerate}[label=\Alph*.]
    \item Decisions that are made via a vote outlined in §\ref{sect:decision-making:voted} can be overridden by a majority vote of members of the Club Administration.
    \item Decisions that are made via executive order can be overridden by a 2/3 majority of Club Administration members at a regular Club Administration meeting.
\end{enumerate}

% Article: advisor
\section{Organization Advisor} \label{art:advisor}

\subsection{Qualifications} \label{sect:advisor:qualifications}
The position of Organization Advisor shall be given to a faculty member of the University.

\subsection{Role} \label{sect:advisor:role}
The role of the Club Advisor is to oversee club activities, to ensure the Club Administration is working towards the Club Mission, to approve expenses (if required), and to assist the club in fulfilling the Club Mission.

% Article: funding
\section{Organization Funds} \label{art:funds}

\subsection{Sources for Funding} \label{sect:funds:sources}
Funding for the Club shall come from the budget allocated by the Center for Campus Life or though fundraisers, donations, or revenue from events hosted by the Club.

\subsection{Management of Club Funds} \label{sect:funds:management}
Management of Club funds shall be the responsibility of the Treasurer.
The President and Vice President have overriding authority on the use of Club funds.

\subsection{Spending of Club Funds} \label{sect:funds:spending}
Club funds shall be spent only for projects, events, and supplies directly in support of the Club Mission outlined in Article \ref{art:mission}.

% Article: diminshed (commonly referred to as reduced, as that is the internal name) club state
\section{Diminished Club State} \label{art:reduced}

\subsection{Entrance of Diminished Club State} \label{sect:reduced:entrance}
The club will enter a diminished state if any one of the following occurs:

\begin{itemize}
    \item FIRST cancels/suspends/postpones/etc. all events in the United States.
        Once events are resumed, the club will automatically return to a normal state (assuming that there are no other constraints holding the club in a diminished state).
    \item The University cancels/suspends/postpones/etc. all club meetings or public events.
        Once club meetings are resumed, the club will automatically return to a normal state (assuming that there are no other constraints holding the club in a diminished state).
    \item The Club Administration unanimously votes to put the club into a diminished state.
        The Club Administration will then hold a majority vote to return the club to a normal state.
\end{itemize}

The transition into a diminished club state shall include announcements posted on all club communication channels and social media platforms within 24 hours of entering a diminished state.
These announcements should inform club members and the general public about the changes that will be made within the club, which events and meetings are cancelled, and how future meetings will be conducted.

\subsection{Definition of Diminished Club State} \label{sect:reduced:definition}
While the club is in a diminished state, only essential activities are performed (where an essential activity is something to keep the club in good-standing with the Center for Campus Life or the University or one of the following processes defined in this document: elections and impeachment).
Other club activities should be adapted to fit the circumstances placed on the club.

\subsubsection{Membership Requirements During a Diminished Club State} \label{subsect:reduced:membership}
All members retain their current membership status and the attendance rate defined in Section \ref{sect:membership:active} and attendance requirements in Section \ref{sect:membership:inactive} and Section \ref{sect:membership:dismissal} are waived.
Members can become Active Members if they are active in the club communication channels for four weeks.
Dues may still be required, as defined in Section \ref{sect:membership:dues}, but a standardized online payment system (such as CampusGroups) must be used to accept payment.

\subsubsection{Meetings During a Diminished Club State} \label{subsect:reduced:meetings}
All club meetings and events must be adapted to the constraints placed on the club (i.e. all meetings cancelled, online meetings, etc.).
The changes to club meetings must be discussed within the Club Administration and announced when the club enters a diminished state (see Section \ref{sect:reduced:entrance}).

\subsubsection{Elections During a Diminished Club State} \label{subsect:reduced:cadmin}
The election process in Section \ref{subsect:cadmin:selection:election} should be utilized with the following exceptions:

\begin{itemize}
    \item Elections should attempt to meet the deadlines proposed, but elections can be held until the last day of the semester, if necessary
    \item The President must make announcements to the club at-large at all points during the election stating the current status of the election (i.e. the timeline for the election, nominees for any given position, time at which voting opens, etc.)
    \item Nominations for the electoral process must be accepted during week before the election process starts and must be announced in a club-wide communication channel or given to the President who will announce it to the club at-large within a reasonable time frame
    \item Speeches can be delivered via alternate means (i.e. video, online meeting, text, etc.) and each nominee should make one announcement in a club-wide communication channel with their speech 
    \item The President will accept digital ballots, sent via a private message on any club communication channel, for 24 hours per position
\end{itemize}

\subsubsection{Impeachment During a Diminished Club State} \label{subsect:reduced:impeachment}
The impeachment process in Section \ref{subsect:cadmin:removal:impeachment} should be utilized with the following exceptions:

\begin{itemize}
    \item Petitions to initiate the impeachment process shall be completed online
    \item If the club is not holding regular meetings, then the accuser must find an acceptable meeting time between all parties (accuser, accused, Club Advisor, and a majority of Active Members) and setup online meetings
\end{itemize}

\subsubsection{Decision Making During a Diminished Club State} \label{subsect:reduced:decisions}
If a voted decision is to be made, then the vote can either be performed using online voting tools or by digital ballots privately messaged to the President; the President is responsible for choosing a voting method.

\subsubsection{Amending the Constitution During a Diminished Club State} \label{subsect:reduced:amending}
The amendment process in Article \ref{art:amending} should be utilized with the following exceptions:

\begin{itemize}
    \item Amendment proposals need to be distributed to all club members digitally using internal club-wide communication channels
    \item Voting on the amendment will commence one week after it was officially proposed and will be open for 24 hours.
        Once the amendment is open to voting, the President must make an announcement to the club at-large informing them of the vote.
    \item The President will accept digital votes, sent via a private message on any club communication channel, from any Active Member (defined in Section \ref{sect:membership:active}) that has read the proposal
\end{itemize}

% Article: amendments
\section{Guidelines for Amending} \label{art:amending}

\subsection{Proposal for Amendments} \label{sect:amending:proposal}
An amendment to this constitution may be proposed by any member in good standing as outlined by Article \ref{art:membership}.
The proposition must be presented at a regular meeting.
Paper copies of the proposition must be make available to all members who prefer a printed copy and available to all members digitally through internal club communication channels.

\subsection{Voting on Amendments}  \label{sect:amending:voting}
\begin{enumerate}[label=\Alph*.]
    \item The vote for an amendment shall be made at the next meeting after the amendment is proposed.
    \item In order to be eligible to vote, a member must be an Active Member as defined in Section \ref{sect:membership:active}, and have attended the meeting where the amendment was proposed.
    \item Voting for the amendment shall be through secret ballot which is collected and counted by the President.
\end{enumerate}

\subsection{Passage of an Amendment} \label{sect:amending:passage}
\begin{enumerate}[label=\Alph*.]
    \item Passage of a proposed amendment required a 2/3 vote by present eligible voters, a quorum being present.
        In the event of a tied vote the president shall break the tie.
    \item Before the proposed amendment shall become operational, it must be approved by the Advisor(s).
\end{enumerate}

% Article: enabling clause
\section{Enabling Clause} \label{art:enabling}
This constitution was voted on and put into effect on the 3\textsuperscript{rd} of April, two-thousand and eighteen. \\

\noindent Amendment \#1, Formal Name Change, was voted on and put into effect on the 30\textsuperscript{th} of April, two-thousand and nineteen. \\

\noindent Amendment \#2, Membership Definitions, was voted on and put into effect on the 30\textsuperscript{th} of April, two-thousand and nineteen. \\

\noindent Amendment \#3, Changes to Club Administration Eligibility, was voted on and put into effect on the 30\textsuperscript{th} of April, two-thousand and nineteen. \\

\noindent Amendment \#4, Modification to Sources for Funding, was voted on and put into effect on the 30\textsuperscript{th} of April, two-thousand and nineteen.

\end{document}
