\documentclass[english,11pt]{article}

%% packages
\usepackage[utf8]{inputenc} % set the input encoding to UTF-8
\usepackage{fancyhdr} % header and footer commands
\usepackage[margin=.6in]{geometry} % sets the margins
\usepackage[useregional]{datetime2} % \today
\usepackage{enumitem} % used for ordered lists
\usepackage[T1]{fontenc} % use Type 1 versions of Computer Modern
\usepackage{lmodern} % use high-quality fonts

%% setup the footers
\pagestyle{fancy}
\fancyhf{}
\lfoot{RIT FIRST Alumni Association}
\rfoot{Page \thepage}
\renewcommand{\headrulewidth}{0pt} % disable header line
\renewcommand{\footrulewidth}{1pt} % set thin footer line

%% Redefine sections to be "Articles"
% see https://tex.stackexchange.com/questions/136527/section-numbering-without-numbers
\makeatletter
% we use \prefix@<level> only if it is defined
\renewcommand{\@seccntformat}[1]{%
  \ifcsname prefix@#1\endcsname
    \csname prefix@#1\endcsname
  \else
    \csname the#1\endcsname\quad
  \fi}
% define \prefix@section
\newcommand\prefix@section{Article \thesection: }
\newcommand\prefix@subsection{Section \thesubsection: }
\makeatother

%% setup the draft watermark, if applicable
\input{watermark.tex} % autogenerated file that will contain watermark info if it is needed

%% set initial page numbering to roman numerals
\pagenumbering{roman}

%% start the document
\begin{document}

%% build the title page
\title{Constitution for the FIRST Robotics Club of the Rochester Institute of Technology}
\author{
    Brett Johnson, Lorenzo Romero, Dustyn Crowley, Connor Henley, Alexander Kneipp, et al.  % your name here (before et al.) <<<<<
    \\
    \texttt{[btj3571, llr4395, dtc3536, cxh1451, ahk8565, first]@rit.edu}
}
\date{} % don't put the date here

% setup the footer for the title page
\fancypagestyle{plain}{
    \fancyhf{}
    \input{buildinfo.tex} % take the lfoot from the autogenerated build info file
    \rfoot{\today}
    \renewcommand{\footrulewidth}{0pt} % disable footer line
}

\maketitle
\newpage % finish title page
\pagenumbering{arabic} % set to normal numerical (arabic) page numbers

% notes:
%   * it is best practice to have a sentence per line, as git will work better
%   * try to use the existing packages/styling

%% start constitution
\section*{Preface}
This document is the official constitution of the FIRST Robotics club, herein referred to as the Club, at the Rochester Institute of Technology, herein referred to as the University.
The guidelines and policies laid out in the following text shall be adhered to, to the best of their ability, by all members of Club Administration, and members of the club at large.

% Article: name and formal stuff
\section{Formal Name} \label{art:formal-name}
The name of this organization shall be the FIRST Robotics Club for all business within the University.
For business outside of the university, the organization shall do business as the RIT FIRST Alumni Association (formally) or RIT FIRST (informally).

% Article: mission
\section{Organization Mission} \label{art:mission}
The mission of the Club is to promote FIRST (For Inspiration and Recognition of Science and Technology) robotics, support and foster interest in robotics programs in both the local community and in general, and provide support for local teams, which includes hosting events, running programs to help teams grow, and maintaining their interest in science and technology.

% Article: membership
\section{Membership} \label{art:membership}

\subsection{Requirements for Membership} \label{sect:membership:requirements}
Membership is open to all regularly enrolled undergraduate and graduate students, alumni, faculty, staff, and part time students in good standing with the University regardless of their previous experience in FIRST or a lack thereof.

\subsection{Definitions of Membership Terms} \label{sect:membership:definitions}
The following definitions are used when defining membership:

\begin{itemize}
    \item attendance - attendance is counted on a weekly basis; members must attend one or more club sanctioned meeting or activity during a week to be counted as present for that week
    \item attendance rate - the number of weeks that the member was marked present divided by the number of weeks since attendance was first taken for this member during the current semester
\end{itemize}

\subsection{Definition of Active Membership} \label{sect:membership:active}
Active Members shall be those who meet the following criteria:
\begin{enumerate}[label=\alph*.]
    \item regularly enrolled undergraduate or graduate student, including those on local co-ops if they so choose,
    \item meets the following attendance requirements:
    \begin{itemize}
        \item maintains a 70\% attendance rate
        \item the member was an Active Member or held temporary Inactive status in the previous semester or has attended at least four meetings during the current semester,
    \end{itemize}
    \item paid club dues to date (if required),
    \item met the qualifications set forth in Section \ref{sect:membership:requirements} along with those of the Center for Campus Life.
\end{enumerate}

\noindent Active Members may attend meetings, participate in discussions, vote in elections, and hold office.

\subsection{Definition of Associate Membership} \label{sect:membership:associate}
Associate members shall be those who have interest in the Club but no interest in or availability to participate.
Associate members may attend meetings and discussions, but shall have no power to vote in the organization and cannot hold office.

\subsection{Definition of Inactive Membership} \label{sect:membership:inactive}
An Active Member shall be declared Inactive if they have missed the required number of meetings or more, as defined in Section \ref{sect:membership:active}.
Members can request temporary Inactive status for the duration of the semester if they will be unable to attend regular club meetings (e.g. member will be on co-op or cannot make meetings due to a class, etc.).
Requests for temporary Inactive status must be submitted to the President or Vice President and will be approved at the following Club Administration regular meeting.
Inactive members share the same privileges and restrictions as Associate Members.

\subsection{Dues} \label{sect:membership:dues}
Dues may be set by the Club Administration at any point to provide additional funding for club activities.
If dues are set, all Active Members must be notified three regular meetings in advance.

\subsection{Member Resignation} \label{sect:membership:resignation}
A member may resign by submitting the intention, in writing, to the President, which becomes effective as soon as it is submitted.

\subsection{Dismissal from Club Roster} \label{sect:membership:dismissal}
A member who was not present for eight weeks within a semester without notification may be dropped from the Club roster and removed from all club resources (dismissed).

% Article: advisor
\section{Club Advisor} \label{art:advisor}

\subsection{Qualifications} \label{sect:advisor:qualifications}
The position of Club Advisor shall be given to a staff/faculty member of the University that meets the qualifications set by the Center for Campus Life.

\subsection{Role} \label{sect:advisor:role}
The role of the Club Advisor is to oversee club activities, to ensure the Club Administration is working towards the Club Mission, to approve expenses (if required), and to assist the Club in fulfilling the Club Mission.

\subsection{Term of Service} \label{sect:advisor:term}
The Club Advisor shall serve a one-year term starting on the last day of the academic year.
The Club Administration shall decide the next Club Advisor and invite them to the position at least one month before the end of the academic year.

% Article: meetings
\section{Meetings} \label{art:meetings}

\subsection{Decision of Meeting Time} \label{sect:meetings:decision}
Meeting times shall be decided by members of the Club Administration (with consideration of the availability of all members who respond to a request for available times) and must be done three weeks before the beginning of the next semester.
General meetings must be held weekly (preferably on a weekday) and will consist of an update on Club activities, volunteering, and mentoring, discussions or decisions on future Club activities and events, and when applicable, elections and impeachments.
Project meetings must be held weekly (preferably on a Saturday or Sunday) and will consist of time for Club members to work on the Club project.

\subsection{Announcement of Meeting Time} \label{sect:meetings:announcement}
Meeting times shall be posted to all Club communication channels and social media platforms before the start of the next academic term.

\subsection{Attendance of Meetings} \label{sect:meetings:attendance}
Attendance at any regular meeting is open to persons who meet the criteria defined in Section \ref{sect:membership:requirements}.

% Article: decision making
\section{Decision Making Process} \label{art:decision}

\subsection{Definitions of Decision Making Terms} \label{sect:decision:definitions}
The following definitions are used in the decision making processes:

\begin{itemize}
    \item coordinator - the individual responsible for running the election process
    \item null vote - a blank vote, a vote for `null', `abstain', `none of the above', etc. that shows dissatisfaction with the nominees
    \item protested election - the most amount of ballots cast are null votes
\end{itemize}

\subsection{Process for a Standing Decision} \label{sect:decision:standing}
The following process shall be used for a standing decision:

\begin{enumerate}
    \item the coordinator must announce the options which can be voted for
    \item the coordinator must ask each eligible voting member to stand and announce their vote
    \item[--] if the voting member elects to do so, then they may disclose their vote to the coordinator only
    \item the coordinator must then count the votes and the option with the most amount of non-null votes is considered the winning result of the decision
\end{enumerate}

\subsection{Process for a Voted Decision} \label{sect:decision:voted}
The following process shall be used for a voted decision:

\begin{enumerate}
    \item the coordinator must announce the options which can be voted for and a brief description of the voting process
    \item paper ballots must be distributed to all eligible voting members present
    \item members must select one option and write their decision on the ballot
    \item after members have finished voting, the coordinator shall then collect the ballots from members
    \item the coordinator shall then count the ballots (discarding all invalid ballots) and the option with the most amount of votes is considered the winning result of the decision
\end{enumerate}

\subsubsection{Process for an Ordered Voted Decision} \label{subsect:decision:voted:ordered}
In the case in which an ordered vote is needed, the process for a voted decision should be used with the following exceptions:

\begin{itemize}
    \item members must write their rank of some or all of the options on the ballot
    \item the coordinator shall count the votes and decide the winner using an instant run-off process:
    \begin{enumerate}
        \item \label{order:ordered-vote:start} the highest-ranking non-eliminated option is counted from each ballot
        \item if an option has greater than 1/2 of the vote, then this option is the winner
        \item if not, then the lowest-ranking option is eliminated from all ballots and the process restarts at \#\ref{order:ordered-vote:start}
    \end{enumerate}
\end{itemize}

\subsubsection{Overriding a Voted Decision} \label{subsect:decision:voted:override}
Decisions that are made via a voted decision can be overridden by a majority vote of members of the Club Administration if voted on at a regular Club Administration meeting.

\subsection{Process for an Election} \label{sect:decision:election}
The following process shall be used for an election:

\begin{enumerate}
    \item \label{order:election:start} the coordinator must announce the current position being elected and the nominees for the given position
    \item paper ballots must be distributed to all Active Members present
    \item members must select one option and write their decision on the ballot
    \item after members have finished voting, the coordinator shall then collect all the secret ballots from members
    \item the coordinator shall then count the ballots and select the winner of the election, which is defined as the member with the most amount of votes
    \item \label{order:election:protested} if a protested election occurs, then a session of nominations will start immediately, where any Active Member can be nominated for the contested role.
        Next, each nominee will be eligible to give a speech if they have not already given one or to withdraw from the election.
        Finally, the process restarts at \#\ref{order:election:start}.
\end{enumerate}

\subsection{Process for a Formal Motion} \label{sect:decision:formal-motion}
The following process shall be used for a formal motion:

\begin{enumerate}
    \item the coordinator must announce the motion that is being voted on
    \item paper ballots must be distributed to all eligible voting members
    \item members must select `yea' or `nay' and write their decision on the ballot
    \item after members have finished voting, the coordinator shall then collect all the secret ballots from members
    \item the coordinator shall then count the ballots and the motion should be considered passed if at least 2/3 of the votes are `yea' votes
\end{enumerate}

\subsection{Proxy Ballot} \label{sect:decision:proxy}
A proxy ballot is a paper ballot that was completed by a member and designed to be submitted in a scheduled vote, yet submitted by a different member.
Proxy ballots must denote the vote that it should be used in and authorized by the original member via signature.
Optionally, proxy ballots may have an authorization code written on them, which can be distributed by the coordinator of the decision.
Proxy ballots are valid in standing decisions, voted decisions, ordered voted decisions, elections, and certain formal motions.
Proxy ballots may be resubmitted in the case of a protested election (see Section \ref{sect:decision:election}\#\ref{order:election:protested}).

\subsection{Process for an Executive Order} \label{sect:decision:executive-order}
Executive Orders are decisions made without a vote by the general club nor the Club Administration.
Executive Orders take effect immediately after their announcement to the Club Administration.

\subsubsection{Restrictions on Executive Orders} \label{subsect:decision:executive-order:restrictions}
An Executive Order must comply with the following restrictions:

\begin{itemize}
    \item it must follow the rules and regulations laid out in this document and cannot override or modify anything in this document without going through the amendment process (see Article \ref{art:amending})
    \item it cannot modify the club roster, including the status of members, with the exception of ad hoc positions (see Section \ref{sect:cadmin:adhoc})
\end{itemize}

\subsubsection{Overriding an Executive Order} \label{subsect:decision:executive-order:override}
Decisions that are made via Executive Order can be overridden by vote with a 2/3 majority of Club Administration members at a regular Club Administration meeting.

% Article: definition of club administration and how it functions
\section{Club Administration Roles and Selection} \label{act:cadmin}

\subsection{President} \label{sect:cadmin:president}
The President is the ultimate presiding figure for the Club.
The responsibilities and duties of the President are as follows:

\begin{enumerate}[label=\Alph*.]
    \item Preside over all club meetings and Club Administration meetings.
    \item Exercise general supervision over all club activities.
    \item Give final approval on any club expense.
    \item Support club administrators and club members in fulfilling the Club Mission Statement.
    \item Ensure that appropriate members of Club Administration are financially certified with the club office.
    \item Has the ability to make Executive Orders, as defined in Section \ref{sect:decision:executive-order}.
\end{enumerate}

\subsection{Vice President} \label{sect:cadmin:vicepresident}
The responsibilities and duties of the Vice President are as follows:

\begin{enumerate}[label=\Alph*.]	
    \item Assume all responsibilities and duties of the President during the temporary absence of the President.
    \item Shall be responsible to maintain open channels of communication between the Club and mentored teams/organizations.
    \item Oversee all ad hoc administrative roles.
    \item Shall be responsible for maintaining our club inventory with the College of Engineering.
\end{enumerate}

\subsection{Treasurer} \label{sect:cadmin:treasurer}
The responsibilities and duties of the Treasurer are as follows:

\begin{enumerate}[label=\Alph*.]
    \item Submit a monthly report of club finances at Club Administration meetings.
    \item Fill out, at the request of the President or Vice President, Expense Approval Forms (EAFs) and any other required financial documents for club events and projects.
    \item Accept all requests for purchases or reimbursements from members of the Club, and get approval from the President or Vice President.
    \item Assume the duties and responsibilities of the Vice President in their absence.
    \item Supervise all financial transactions of the organization and maintain all financial records.
    \item Be responsible for communication with the Director of Finance of the Center for Campus Life.
\end{enumerate}

\subsection{Media and Communications Manager} \label{sect:cadmin:mediacomms}
The responsibilities and duties of the Media and Communications Manager are as follows:

\begin{enumerate}[label=\Alph*.]
    \item Update, or oversee updates to, news sections on the Club website.
    \item Coordinate, at the request of the President or Vice President, distribution of flyers and other club promotional materials.
    \item Act as a liaison between the Club and FIRST community regarding club events, club presence at local competitions, and mentoring.
    \item Actively seek out, at the request of the President or Vice President, new opportunities for mentor-ship and club involvement in the local FIRST community.
    \item Maintain an active social media presence via the Club social media accounts, such as periodic posts about recent meetings and project milestones, and reminders about upcoming meetings and events.
\end{enumerate}

\subsection{Ad Hoc Administrative Positions} \label{sect:cadmin:adhoc}
The following positions exist as needed, and may be filled at the will of the President by a club member, including one already in a Club Administration position, should they accept the appointment:

\begin{enumerate}[label=\Alph*.]
    \item {\large Data Steward} \\
            The Data Steward shall be a member of the Club who is well versed in the operations of the Club's document and resource storage system and be able to effectively administer said system.
            The Data Steward shall also maintain the Club's project management system, and collaboration sites.
    \item {\large Webmaster} \\
            The Webmaster shall maintain the web presence of the Club, primarily through the official club website (ritfirst.org).
            They shall fulfill requests of the President and Vice President regarding content posted on the official Club website.
            It will be the responsibility of the Webmaster to assist club members in getting relevant information available on the web, including project updates, meeting times, and other club resources.
            The Webmaster shall also maintain the Club's communication channels.
    \item {\large Imagine RIT Project Manager} \\
            The Imagine Project Manager shall assume ultimate responsibility for the completion of the Imagine project in a timely manner, planning and logistics, processing orders related to the project, and managing project resources.
    \item {\large FRC Robot Coordinator} \\
            The FRC Robot Coordinator shall oversee any projects regarding FRC robots obtained from local schools.
            They shall ensure that all components and parts obtained are inventoried, and project resources are managed properly.
    \item {\large Meeting Secretary} \\
            The Meeting Secretary shall compile meeting minutes and send them out the Club at-large via email.
            In the event that the Meeting Secretary will be absent, they must notify the President and suggest another member to fulfil the role.
\end{enumerate}

\subsection{Selection of Club Administration Roles} \label{sect:cadmin:selection}

\subsubsection{Qualifications for Candidate} \label{subsect:cadmin:selection:qualifications}
Any candidate for a Club Administration role must be in good academic standing as defined by the University.
The candidate must have been an Active Member in good standing in the Club during the previous academic term, not including any summer terms.

\subsubsection{Process for Election} \label{subsect:cadmin:selection:election}
An election must take place at least 2 meetings prior to the last Club meeting for that academic year. The following guidelines must be followed:

\begin{enumerate}[label=\alph*.]
    \item Nominations shall be made for members from an open floor at the meeting prior to the election.
          A person may be nominated for more than one position.
    \item Members who will be on co-op during their term may not be nominated for the role of President, unless there are no eligible members that accept nominations, or for the role of Vice President, unless the Club Administration unanimously approves or there are no eligible members that accept nominations.
          However, both the President and Vice President cannot be on co-op at the same time.
          Members who are on co-op during nominations and will be eligible for a Club Administration position when they return, can be nominated and submit their speech to the President, who will distribute it (electronically or physically) to all eligible voting club members.
    \item The order of election shall be President, Vice President, Treasurer, and Media \& Communications
    \item Voting shall follow the election process defined in Section \ref{sect:decision:election} where the President is the coordinator of the vote.
    \item Inductions shall not take place until after a meeting between the current Club Administration and incoming Club Administration.
    \item At the aforementioned meeting, the incoming Club Administration shall be presented with all information pertaining to the current state of the Club.
        This meeting should also include all important persons that frequently deal with the Club, such as the Club Advisor.
\end{enumerate}

\subsubsection{Term of Service} \label{subsect:cadmin:selection:terms}
Members of Club Administration shall serve a one-year term, starting on last day of academic year in which they were elected.

\subsection{Removal of a Club Administration Member} \label{sect:cadmin:removal}

\subsubsection{Impeachment of a Club Administration Member} \label{subsect:cadmin:removal:impeachment}

\begin{enumerate}[label=\alph*.]
    \item Any member of the Club Administration may be impeached, ad hoc roles may be dismissed by order of the President.
    \item Impeachment may be initiated by petition, in writing, by 25\% or more of the Club's Active Members, when presented to the floor in a regular meeting.
    \item The accuser is responsible for communicating with the Club Advisor and organizing their attendance at the next regular meeting.
        The highest-ranking non-accused Club Administration member must notify the Center for Campus Life staff of an impeachment trial.
    \item At the next regular meeting, the accuser (members signing the petition) and the accused (club administrator) shall present their case to the members.
        This meeting shall be presided over by the Club Advisor.
    \item Both cases will have the opportunity to present an introduction, an argument, and a conclusion (in this order).
        The impeachment trial will start with the accuser presenting their introduction, then the accused will present their introduction; this alternating process will continue for the arguments and conclusions.
        Each case is limited to a total of 20 minutes of presentation time, each group can use it as they see fit.
    \item Voting shall follow the formal motion process defined in Section \ref{sect:decision:formal-motion} where the eligible voting members are all Active Members present and the Club Advisor, the coordinator is the Club Advisor, and no proxy ballots are accepted. 
    \item Conviction of impeachment shall cause immediate removal from office, loss of all privileges thereof, and suspension of all responsibilities and duties that role entails.
    \item The filling of a position vacated by impeachment shall be by a special election held using the election process defined in Section \ref{sect:decision:election} where the coordinator is the highest-ranking Club Administration member.
        Nominations for this election shall be held at the meeting following impeachment.
        The special election shall be held at the meeting following nominations.
\end{enumerate}

\subsubsection{Resignation of a Club Administration Member} \label{subsect:cadmin:removal:resignation}

\begin{enumerate}[label=\alph*.]
    \item Any member of the Club Administration may resign at their discretion.
    \item The resignation letter must be presented to the Club in the same manner as a member resignation as outlined in Section \ref{sect:membership:resignation}.
    \item The filling of a vacancy as a result of a resignation shall be by a special election held using the election process defined in Section \ref{sect:decision:election} where the coordinator is the highest-ranking Club Administration member.
\end{enumerate}

\subsection{Other Guidelines for Club Administration Members} \label{subsect:cadmin:other}

\begin{enumerate}[label=\Alph*.]
    \item The members of Club Administration shall be ranked in the following order: President, Vice President, Treasurer, Media \& Communications Manager.
    \item No member of Club Administration can hold more than one non-ad hoc position at any given time.
    \item All members of Club Administration, except for those who are only filling an ad hoc position, shall be financially certified through the Center for Campus Life.
    \item If a member of the Club Administration goes on co-op or otherwise becomes an Inactive Member during their term, the vacancy created will be filled using the election process defined in Section \ref{sect:decision:election} where the coordinator is the highest-ranking Club Administration member.
          This elected member's term will continue from election until the original member returns from co-op or the term expires.
          This elected member will effectively replace the original member in the Club Administration, except in Club Administration communications channels, where both members may remain.
    \item If the Club is participating in College of Engineering Pizza Sales, then all Club Administration members must complete pizza sale training, as specified by the College of Engineering.
    \item A Club Administration vote shall be performed using the Standing Decision process defined in Section \ref{sect:decision:standing}, where the coordinator is the member calling the vote and the eligible members are Club Administration members.
\end{enumerate}

% Article: funding
\section{Organization Funds} \label{art:funds}

\subsection{Sources for Funding} \label{sect:funds:sources}
Funding for the Club shall come from the budget allocated by the Center for Campus Life or though fundraisers, donations, or revenue from events hosted by the Club.

\subsection{Management of Club Funds} \label{sect:funds:management}
Management of Club funds shall be the responsibility of the Treasurer.
The President and Vice President have overriding authority on the use of Club funds.
Funding should be handled according to Center for Campus Life regulations.

\subsection{Spending of Club Funds} \label{sect:funds:spending}
Club funds shall be spent only for projects, events, and supplies directly in support of the Club Mission outlined in Article \ref{art:mission}.

% Article: diminshed (commonly referred to as reduced, as that is the internal name) club state
\section{Diminished Club State} \label{art:reduced}

\subsection{Entrance of Diminished Club State} \label{sect:reduced:entrance}
The club will enter a diminished state if any one of the following occurs:

\begin{itemize}
    \item FIRST cancels/suspends/postpones/etc. all events in the United States.
        Once events are resumed, the Club will automatically return to a normal state (assuming that there are no other constraints holding the Club in a diminished state).
    \item The University cancels/suspends/postpones/etc. all club meetings or public events.
        Once club meetings are resumed, the Club will automatically return to a normal state (assuming that there are no other constraints holding the Club in a diminished state).
    \item The Club Administration unanimously votes to put the Club into a diminished state.
        The Club Administration will then hold a majority vote to return the Club to a normal state.
\end{itemize}

\noindent The transition into a diminished club state shall include announcements posted on all club communication channels and social media platforms within 24 hours of entering a diminished state.
These announcements should inform club members and the general public about the changes that will be made within the Club, which events and meetings are cancelled, and how future meetings will be conducted.

\subsection{Definition of Diminished Club State} \label{sect:reduced:definition}
While the Club is in a diminished state, only essential activities are performed (where an essential activity is something to keep the Club in good-standing with the Center for Campus Life or the University or one of the following processes defined in this document: elections and impeachment).
Other club activities should be adapted to fit the restrictions placed on the Club.

\subsubsection{Membership Requirements During a Diminished Club State} \label{subsect:reduced:membership}
All members retain their current membership status and the attendance rate defined in Section \ref{sect:membership:active} and attendance requirements in Section \ref{sect:membership:inactive} and Section \ref{sect:membership:dismissal} are waived.
Members can become Active Members if they are active in the Club communication channels for four weeks.
Dues may still be required, as defined in Section \ref{sect:membership:dues}, but a standardized online payment system (such as CampusGroups) must be used to accept payment.

\subsubsection{Meetings During a Diminished Club State} \label{subsect:reduced:meetings}
All club meetings and events must be adapted to the restrictions placed on the Club (i.e. all meetings cancelled, online meetings, etc.).
The changes to club meetings must be discussed within the Club Administration and announced when the Club enters a diminished state (see Section \ref{sect:reduced:entrance}).

\subsubsection{Decision Making During a Diminished Club State} \label{subsect:reduced:decisions}
If a voted decision is to be made, then the vote can either be performed using online voting tools or by digital ballots privately messaged to the President; the President is responsible for choosing a voting method.

\subsubsection{Elections During a Diminished Club State} \label{subsect:reduced:cadmin}
The election process in Section \ref{subsect:cadmin:selection:election} should be utilized with the following exceptions:

\begin{itemize}
    \item Elections should attempt to meet the deadlines proposed, but elections can be held until the last day of the semester, if necessary
    \item The President must make announcements to the Club at-large at all points during the election stating the current status of the election (i.e. the timeline for the election, nominees for any given position, time at which voting opens, etc.)
    \item Nominations for the electoral process must be accepted during the week before the election process starts and must be announced in a club-wide communication channel or given to the President who will announce it to the Club at-large within a reasonable time frame
    \item Speeches can be delivered via alternate means (i.e. video, online meeting, text, etc.) and each nominee should make one announcement in a club-wide communication channel with their speech 
    \item The President will accept digital ballots, sent via a private message on any club communication channel, for 24 hours per position
\end{itemize}

\subsubsection{Impeachment During a Diminished Club State} \label{subsect:reduced:impeachment}
The impeachment process in Section \ref{subsect:cadmin:removal:impeachment} should be utilized with the following exceptions:

\begin{itemize}
    \item Petitions to initiate the impeachment process shall be completed online
    \item If the Club is not holding regular meetings, then the accuser must find an acceptable meeting time between all parties (accuser, accused, Club Advisor, and a majority of Active Members) and setup online meetings
\end{itemize}

\subsubsection{Amending the Constitution During a Diminished Club State} \label{subsect:reduced:amending}
The amendment process in Article \ref{art:amending} should be utilized with the following exceptions:

\begin{itemize}
    \item Amendment proposals need to be distributed to all club members digitally using internal club-wide communication channels
    \item Voting on the amendment will commence one week after it was officially proposed and will be open for 24 hours.
        Once the amendment is open to voting, the President must make an announcement to the Club at-large informing them of the vote.
    \item The President will accept digital votes, sent via a private message on any club communication channel, from any Active Member (defined in Section \ref{sect:membership:active}) that has read the proposal
\end{itemize}

% Article: amendments
\section{Guidelines for Amending} \label{art:amending}

\subsection{Proposal for Amendments} \label{sect:amending:proposal}
An amendment to this constitution may be proposed by any member in good standing as outlined by Article \ref{art:membership}.
The proposition must be presented at a regular meeting.
Paper copies of the proposition must be made available to all members who prefer a printed copy and available to all members digitally through internal club communication channels.

\subsection{Acceptance of Amendments} \label{sect:amending:acceptance}
\begin{enumerate}[label=\Alph*.]
    \item The vote for an amendment shall be made at the next meeting after the amendment is proposed.
    \item Each amendment must be voted on as a Formal Motion, as defined in Section \ref{sect:decision:formal-motion}, where a voting member must be an Active Member (defined in Section \ref{sect:membership:active}) that has read the amendment proposal and the coordinator of the vote is the President.
    \item Before the proposed amendment shall become operational, it must be approved by the Club Advisor.
\end{enumerate}

% Article: adherence to University Policy
\section{Adherence to University Policy} \label{art:university-policy}

\subsection{Anti-Hazing} \label{sect:university-policy:anti-hazing}
\underline{\textbf{Per the RIT Hazing Policy}} (\textit{RIT Student Conduct Process; IV. RIT Code of Conduct; 14. Hazing/Failure to Report Hazing}) \\

\noindent Hazing/Failure to Report Hazing.
Behavior, regardless of intent, which endangers the emotional, or physical health and safety of a Student for the purpose of membership, affiliation with, or maintaining membership in, a group or Student Organization.
Hazing includes any level of participation, such as being in the presence, having awareness of hazing, or failing to report hazing.
Examples of hazing include, but are not limited to, beating or branding, sleep deprivation or causing excessive fatigue, threats of harm, forcing or coercing consumption of food, water, alcohol or other drugs, or other substances, verbal abuse, embarrassing, humiliating, or degrading acts, or activities that induce, cause or require the Student to perform a duty or task which is not consistent with fraternal law, ritual or policy or involves a violation of local, state or federal laws, or the RIT Code of Conduct. \\

\noindent \underline{\textbf{NY State Hazing Law}} \\
\noindent \textsection 120.16 Hazing in the first degree.
A person is guilty of hazing in the first degree when, in the course of another person's initiation into or affiliation with any organization, he intentionally or recklessly engages in conduct which creates a substantial risk of physical injury to such other person or a third person and thereby causes such injury.
Hazing in the first degree is a class A misdemeanor. \\

\noindent \textsection 120.17 Hazing in the second degree.
A person is guilty of hazing in the second degree when, in the course of another person's initiation or affiliation with any organization, he intentionally or recklessly engages in conduct which creates a substantial risk of physical injury to such other person or a third person.
Hazing in the second degree is a violation. 

\subsection{Anti-Discrimination Clause} \label{sect:university-policy:anti-discrimination}

This organization shall not discriminate on the basis of sex, race, color, sexual orientation gender identity and gender expression, religion, age marital sate, national origin, disability or veteran status.
This policy will include but is not limited to, recruiting, membership, organization activities or opportunities to hold or run for club office.

\subsection{Statement of Compliance with University Regulations} \label{sect:university-policy:university-regulations}

This organization shall comply with all University and Center for Campus Life policies and regulations, and local, state, and federal laws.

% Article: enabling clause
\section{Enabling Clause} \label{art:enabling}
This constitution was voted on and put into effect on the 3\textsuperscript{rd} of April, two-thousand and eighteen. \\

\noindent Amendment \#1, Formal Name Change, was voted on and put into effect on the 30\textsuperscript{th} of April, two-thousand and nineteen. \\

\noindent Amendment \#2, Membership Definitions, was voted on and put into effect on the 30\textsuperscript{th} of April, two-thousand and nineteen. \\

\noindent Amendment \#3, Changes to Club Administration Eligibility, was voted on and put into effect on the 30\textsuperscript{th} of April, two-thousand and nineteen. \\

\noindent Amendment \#4, Modification to Sources for Funding, was voted on and put into effect on the 30\textsuperscript{th} of April, two-thousand and nineteen. \\

\noindent Amendment \#5, Miscellaneous Changes \#1, was voted on and put into effect on the 28\textsuperscript{th}/29\textsuperscript{th} of April, two-thousand and twenty. \\

\noindent Amendment \#6, Changes to the Decision Making Process, was voted on and put into effect on the 28\textsuperscript{th}/29\textsuperscript{th} of April, two-thousand and twenty. \\

\noindent Amendment \#7, Addition of Diminished Club State, was voted on and put into effect on the 28\textsuperscript{th}/29\textsuperscript{th} of April, two-thousand and twenty.

\end{document}
